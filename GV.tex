\documentclass[useAMS,usenatbib]{mn2e}
\usepackage{footnote}
\usepackage{graphicx}
\usepackage{amsmath}
\usepackage{natbib}
\usepackage{array}
\usepackage{color}
\usepackage{url}
\voffset=-0.5in

%%%%%%%%%%%%%%%%%%%%
% For changes since the old draft describing the peak data - to ones describing runs on single galaxies with GLAMDRING.
%%%%%%%%%%%%%%%%%%%%
\definecolor{titlecol}{rgb}{0,0,0}
\def\changed    {\color{titlecol} }
\definecolor{change}{rgb}{0,0,1}
\def\newchange {\color{change} }
\def\starfpy {\textsc{StarfPy}}



\begin{document}
\title[The Star Formation History of the Green Valley]{Galaxy Zoo: Evidence for Diverse Star Formation Histories through the Green Valley}
\author[Smethurst et al. 2014]{R. ~J. ~Smethurst,$^1$ C. ~J. ~Lintott,$^{1,2}$ B. ~D. ~Simmons,$^{1}$ K. ~Schawinski,$^{3}$ \newauthor  P. ~J. ~Marshall,$^{4,1}$ S. ~Bamford,$^{5}$ L. Fortson,$^{6}$ S. ~Kaviraj,$^{7}$ K.~L.~Masters,$^{8}$ \newauthor  T. ~Melvin,$^{8}$  B. ~Nichol,$^{8}$  R. ~A. ~Skibba,$^{9}$ K. ~W. ~Willett$^{6}$ 
\\ $^1$ Oxford Astrophysics, Department of Physics, University of Oxford, Denys Wilkinson Building, Keble Road, Oxford, OX1 3RH, UK 
\\ $^2$ Adler Planetarium, 1300 S Lake Shore Drive, Chicago, IL, 60605, USA 
\\ $^3$ Institute for Astronomy, Department of Physics, ETH Zurich, Wolfgang-Pauli Strasse 27, CH-8093 Zurich, Switzerland 
\\ $^4$ Kavli Institute for Particle Astrophysics and Cosmology, Stanford University, 452 Lomita Mall, Stanford, CA 95616, USA
\\ $^5$ School of Physics and Astronomy, The University of Nottingham, University Park, Nottingham, NG7 2RD, UK
\\ $^6$ School of Physics and Astronomy, University of Minnesota, 116 Church St SE, Minneapolis, MN 55455, USA
\\ $^7$ Centre for Astrophysics Research, University of Hertfordshire, College Lane, Hatfield, Hertfordshire, AL10 9AB, UK
\\ $^8$ Institute of Cosmology and Gravitation, University of Portsmouth, Dennis Sciama Building, Barnaby Road, Portsmouth, PO1 3FX, UK 
\\ $^9$ Center for Astrophysics and Space Sciences, University of California San Diego, 9500 Gilman Drive, La Jolla, CA 92093, USA
}

\maketitle

\begin{abstract}
Does galactic evolution proceed through the green valley via multiple pathways or as a single population? Motivated by recent results  highlighting radically different evolutionary pathways between early- and late-type galaxies, we present results from a simple Bayesian approach to this problem wherein we model the star formation history of a galaxy and compare the predicted and observed optical and near-ultraviolet colours. We use a novel method to investigate the morphological differences between the most probable values for these parameters for both disc-like and elliptical-like populations of galaxies, by using probabilistic estimates of morphology from Galaxy Zoo\footnotemark[1]. We find that the parameters for the green valley galaxies for both major morphologies follow those of the red sequence but at later times, predicting the build up of the red sequence. The green valley is therefore a transitional population regardless of morphology. However, the rate of this transition depends on the morphology, with three possible routes through the green valley dominated by smooth- (rapid timescales, attributed to major merger), intermediate- (intermediate timescales, attributed to minor mergers and galaxy interactions) and disc-like (slow timescales, attributed to secular evolution) galaxies. {\changed We hypothesise that morphological changes occur in systems which have undergone quenching with a timescale $\tau < 1.5~\rm{Gyr}$, in order for the evolution of galaxies in the green valley to match the ratio of smooth to disc galaxies observed in the red sequence. These rapid timescales are instrumental in the formation of the red sequence at earlier times, however we find that galaxies currently passing through the green valley typically do so at intermediate timescales. The onset of quenching for disc galaxies across the colour-magnitude diagram is distributed evenly over all lookback times, confirming the assumption that whatever is driving the evolution of discs is constant over cosmic time.}

%The majority of elliptical galaxies have undergone transitions with intermediate timescales, comparable to those for minor mergers and galaxy interactions ($\tau \sim 1.5~\rm{Gyr}$), however a small minority ($4.2\%$) have undergone very rapid transitions with timescales comparable to that found for dry major mergers ($\tau \la 0.4~\rm{Gyr}$). This rapid quenching is not seen in disc-like galaxies, which have largely quenched on intermediate timescales, however the majority were found to transition at slower at timescales comparable to those for secular evolution mechanisms ($\tau \sim 2.5~\rm{Gyr}$). A minority of disc galaxies ($3.9\%$) quenched at the earliest times ($t \sim 3~\rm{Gyr}$) and have slow quenching timescales are therefore only just beginning to reach the red sequence at the current observational epoch.
\end{abstract}

\\
\footnotetext[1]{This investigation has been made possible by the participation of more than 250,000 users in the Galaxy Zoo project. Their contributions are individually acknowledged at http://www.galaxyzoo.org/volunteers.aspx}

\section{Introduction}
Previous large scale surveys of galaxies have revealed a bimodality in the colour-magnitude diagram (CMD) with two distinct populations; one at relatively low mass, with blue optical colours and another at relatively high mass, with red optical colours \citep{Baldry04, Baldry06, Willmer06, BLB08, Brammer09}. These populations were dubbed the `blue cloud' and `red sequence' respectively. The Galaxy Zoo project \citep{Lintott11}, which incorporated morphological classifications for a million galaxies, revealed that this bimodality is not entirely morphology driven \citep{Strat01, Bamford09, Skibba09}, detecting spiral galaxies in the red sequence \citep{Masters10} and elliptical galaxies in the blue cloud \citep{Sch09}.  

\begin{figure*}
\centering{
\includegraphics[width=\textwidth]{mosaic_disc_fraction_z_0-07_0-075_(2).pdf}}
\caption{{\newchange Randomly selected SDSS \emph{gri} composite images showing the continuous probabilistic nature of the Galaxy Zoo sample from a redshift range $0.070 < z < 0.075$. The debiased disc vote fraction (see \citealt{GZ2}) for each galaxy is shown. The scale for each image is $0.099~\rm{arcsec/pixel}$.}}
\label{mosaic}
\end{figure*}

The sparsely populated colour space between these two populations, the so-called `green valley', provides clues to the nature and duration of galaxies' transitions from blue to red. This transition must occur on rapid timescales, otherwise we would find an accumulation of galaxies residing in the green valley, rather than an accumulation in the red sequence as is observed \citep{Arnouts07, Martin07}. Green valley galaxies have therefore long been thought of as the `crossroads' of galaxy evolution, a transition population between the two main galactic stages of the star forming blue cloud and the `dead' red sequence \citep{Bell04, Wyder07, Schim07, Martin07, Faber07, Mendez11, Gonc12, Sch2014, Pan14}. 


The intermediate colours of these green valley galaxies have been interpreted as evidence for recent quenching (suppression) of star formation \citep{Salim07}. Star forming galaxies are observed to lie on a well defined mass-SFR relation, however quenching a galaxy causes it to depart from this relation (\citealt{Noeske07, Peng}; see Figure~\ref{sfr_mass_sub})


By studying the galaxies which  have just left this mass-SFR relation, we can probe the quenching mechanisms by which this occurs. There have been many previous theories for the initial triggers of these quenching mechanisms, including negative feedback from AGN \citep{Sch07}, mergers \citep{Darg10a}, supernovae winds \citep{MFB12} and secular evolution \citep{Masters10, Masters11}. By investigating the \emph{amount} of quenching that has occurred between the blue cloud, green valley and red sequence (the three populations) we can apply some constraints to these theories. 

We have been motivated by a recent result suggesting two contrasting evolutionary pathways through the green valley by different morphological types (\citealt{Sch2014}, hereafter S14), specifically that late-type galaxies quench very slowly and form a nearly static disc population in the green valley, whereas early-type galaxies quench very rapidly, transitioning through the green valley and onto the red sequence in $\sim 1$~Gyr \citep{Wong12}. That study used a toy model to examine quenching across the green valley; here we implement a novel method utilising Bayesian statistics (for a comprehensive overview of Bayesian statistics see either \citealt{MacKay} or \citealt{Sivia}) in order to find the most likely model description of the star formation histories of galaxies in the three populations. It also provides a direct comparison with our current understanding of galaxy evolution from stellar population synthesis (SPS, see section ~~\ref{models}) models. 


\begin{table*}
\caption{Table showing the decomposition of the GZ2 sample by galaxy type into the subsets of the colour-magnitude diagram.}
\begin{tabular*}{0.9\textwidth}{r @{\extracolsep{\fill}}cccc}
\hline
\begin{tabular}[c]{@{}c@{}} {\color{white} -} \\ {\color{white} -}  \end{tabular} & All                                                      & Red Sequence                                              & Green Valley                                              & Blue Cloud \\  \hline 
Smooth-like ($p_s > 0.5$)        & \begin{tabular}[c]{@{}c@{}}42453\\ (33.6\%)\end{tabular} & \begin{tabular}[c]{@{}c@{}}17424\\ (13.8\%)\end{tabular}  & \begin{tabular}[c]{@{}c@{}}10687\\ (8.4\%)\end{tabular}   & \begin{tabular}[c]{@{}c@{}}14342\\ (11.3\%)\end{tabular}  \\ 
Disc-like ($p_d > 0.5$)          & \begin{tabular}[c]{@{}c@{}}83863\\ (66.4\%)\end{tabular} & \begin{tabular}[c]{@{}c@{}}10722\\ (8.4\%)\end{tabular}   & \begin{tabular}[c]{@{}c@{}}13257\\ (10.5\%)\end{tabular}  & \begin{tabular}[c]{@{}c@{}}59884\\ (47.4\%)\end{tabular}  \\
Early-type ($p_s \geq 0.8$) & \begin{tabular}[c]{@{}c@{}}10517\\ (8.3\%)\end{tabular}  & \begin{tabular}[c]{@{}c@{}}5337\\ (4.2\%)\end{tabular}    & \begin{tabular}[c]{@{}c@{}}2496\\ (2.0\%)\end{tabular}    & \begin{tabular}[c]{@{}c@{}}2684\\ (2.1\%)\end{tabular}    \\
Late-type ($p_s \geq 0.8$)  & \begin{tabular}[c]{@{}c@{}}51470\\ (40.9\%)\end{tabular} & \begin{tabular}[c]{@{}c@{}}4493\\ (3.6\%)\end{tabular}    & \begin{tabular}[c]{@{}c@{}}6817\\ (5.4\%)\end{tabular}    & \begin{tabular}[c]{@{}c@{}}40430\\ (32.0\%)\end{tabular}  \\ \hline
\textbf{Total}                       & \begin{tabular}[c]{@{}c@{}}\textbf{126316} \\ (100.0\%)\end{tabular}                                                & \begin{tabular}[c]{@{}c@{}}28146 \\ (22.3\%)\end{tabular} & \begin{tabular}[c]{@{}c@{}}23944 \\ (18.9\%)\end{tabular} & \begin{tabular}[c]{@{}c@{}}74226 \\ (58.7\%)\end{tabular} \\\hline
\end{tabular*}
\label{subs}
\end{table*}

Through this approach, we aim to determine the following:
\begin{enumerate}
\item What previous star formation history (SFH) causes a galaxy to reside in the green valley at the current epoch?
%\item Why is the green valley so sparsely populated?
\item Is the green valley a transitional or static population? 
\item If the green valley is a transitional population, how many routes through it are there? 
\item Are there morphology-dependent differences between these routes through the green valley? 
\end{enumerate}

This paper proceeds as follows. Section~\ref{data} contains a description of the sample data, which is used in the Bayesian analysis of an exponentially declining star formation history model, all described in Section~\ref{models}. Section~\ref{results} contains the results produced by this analysis, with Section~\ref{diss} providing a detailed discussion of the results obtained. We also summarise our findings in Section~\ref{conc}. The zero points of all \emph{ugriz} magnitudes are in the AB system and where necessary we adopt the WMAP Seven-Year Cosmological parameters \citep{WMAP} with $(\Omega_m, \Omega_{\lambda}, h) = (0.26, 0.73, 0.71)$. 

\section{Data}\label{data}
\subsection{Multi-wavelength data}\label{multi}
The galaxy sample is compiled from publicly available optical data from the Sloan Digitial Sky Survey (SDSS; \citealt{York00}) Data Release 8 \citep{Aihara11}. Near-ultraviolet (NUV) photometry was obtained from the Galaxy Evolution Explorer (GALEX; \citealt{Martin05}) and was matched with a search radius of $1''$ in right ascension and declination. 

Observed optical and ultraviolet fluxes are corrected for dust reddening using estimates of internal extinction \citep{Oh11} by applying the \citet*{Cardelli89} law. We also adopt k-corrections to $z=0.0$ from the NYU-VAGC \citep{Blanton05, Pad08, BR07} with a typical $u-r$ correction of $\sim 0.05$ mag. Omitting these corrections does not change the results significantly. 

We obtained star formation rates and stellar masses from the MPA-JHU catalog (\citealt{Kauff03, Brinch04}; average values, \textsc{AVG}, corrected for aperture and extinction), which are in turn calculated from the SDSS spectra and photometry. 

We further select a sub-sample with detailed morphological classifications, as described below.

\subsection{Galaxy Zoo 2 Morphological classifications}\label{class}


In this investigation we use visual classifications of galaxy morphologies from the Galaxy Zoo 2\footnote{\url{http://zoo2.galaxyzoo.org/}} citizen science project \citep{GZ2}, which obtains multiple independent classifications for each galaxy image; the full question tree for each image is shown in Figure 1 of \citealt{GZ2}.  

The Galaxy Zoo 2 (GZ2) project consists of $304, 022$ images from the SDSS DR8 (a subset of those classified in Galaxy Zoo 1; GZ1) all classified by \emph{at least} 17 independent users, with the mean number of classifications standing at $\sim42$. The GZ2 sample is more robust than the GZ1 sample and provides more detailed morphological classifications, including features such as bars, the number of spiral arms and the ellipticity of smooth galaxies. It is for these reasons we use the GZ2 sample, as opposed to the GZ1, allowing for further investigation of specific galaxy classes in the future (see Section~\ref{future}). The only selection that was made to the sample was for the removal of  objects considered to be stars or artefacts by the users (i.e. with $p_{star/artefact} ~\geq~ 0.8$). Further to this, we required NUV photometry from the GALEX survey, within which $\sim42\%$ of the GZ2 sample were observed, giving a total sample size of $126, 316$ galaxies. 

\begin{figure}
\centering{
\includegraphics[width=0.45\textwidth]{col_mag_with_GV_pres.pdf}}
\caption{Colour-magnitude diagram for the Galaxy Zoo 2 population showing the definition between the blue cloud and the red sequence from \citet{Baldry04} with the dashed line, as defined in Equation~\ref{eqgv}. The solid lines show $\pm 1\sigma$ either side of this definition; any galaxy within the boundary of these two solid lines is considered a green valley galaxy. {\newchange The overwhelming numbers of blue galaxies in the sample make the green valley hard to see by eye and consequently the green valley as defined by \citet{Baldry04} appears to intersect with the location of the red sequence in the sample. This stresses the importance of using a literature definition for the green valley to ensure comparisons can be made with other works.}}
\label{CMGV}
\end{figure}

The first task asks users to choose whether a galaxy is mostly smooth, is featured and/or has a disc or is a star/artefact. Unlike other tasks further down in the decision tree, every user who classifies a galaxy image will complete this task (others, such as whether the galaxy has a bar, is dependent on a user having first classified it as a featured galaxy). Therefore we have the most statistically robust classifications at this level.

The classifications from users produces a vote fraction for each galaxy (the debiased fractions calculated by \citet{GZ2} were used in this investigation); for example if 80 of 100 people thought a galaxy was disc shaped, whereas 20 out of 100 people thought the same galaxy was smooth in shape (i.e. elliptical), that galaxy would have vote fractions $p_{s} = 0.2$ and $p_{d} = 0.8$. In this example this galaxy would be included in the \emph{`clean'} disc sample ($p_d \geq 0.8$) according to \cite{GZ2} and would be considered a late-type galaxy. {\changed All previous Galaxy Zoo projects have incorporated extensive analysis of volunteer classifications to measure classification accuracy and bias and compute user weightings (for a detailed description of debasing and consistency-based user weightings, see either Section 3 of \citealt{Lintott09} or Section 3 of \citealt{GZ2}). }

{\changed The classifications are highly accurate and provide a continuous scale of morphological features, as shown in Figure~\ref{mosaic}, rather than a simple binary classification separating elliptical and disc galaxies. These classifications allow each galaxy to be considered as a probabilistic object with both bulge and disc components.} For the first time, we incorporate this advantage of the GZ classifications into a large statistical analysis of how elliptical and disc galaxies differ in their SFHs.

\subsection{Defining the Green Valley}\label{defGV}

To define which of the sample of $126, 316$ galaxies were in the green valley, {\changed we looked to previous definitions in the literature defining the separation between the red sequence and blue cloud to ensure comparisons can be made with other works. \citet{Baldry04} used a large sample of local galaxies from the SDSS to trace this bimodality by fitting double Gaussians to the colour magnitude diagram without cuts in morphology.} Their relation is defined in their Equation 11 as:
\begin{equation}\label{eqgv}
C'_{ur}(M_{r}) = 2.06 - 0.244 \tanh \left( \frac{M_r + 20.07}{1.09}\right)
\end{equation}
and is shown in Figure~\ref{CMGV} by the dashed line. {\newchange This ensures that the definition of the green valley is derived from a complete sample, rather than from our sample that is dominated by blue galaxies due to the necessity for NUV photometry.} Any galaxy within $\pm 1\sigma$ of this relationship, shown by the solid lines in Figure~\ref{CMGV}, is therefore considered a green valley galaxy. The decomposition of the sample into red sequence, green valley and blue cloud galaxies is shown in Table~\ref{subs} along with further subsections by galaxy type. This table also defines the definitions we adopt henceforth for early-type ($p_s~ \geq~0.8$), late-type ($p_d~ \geq~0.8$), smooth-like ($p_s~ >~0.5$) and disc-like ($p_d~ >~0.5$) galaxies.

%This ensures that our definition of the green valley is not set by the ratio of blue cloud to red sequence galaxies in our sample, but by the physical location of the green valley.

\section{Models}\label{models}
\subsection{Quenching Models}\label{sfh}
 \begin{figure}
\centering{
\includegraphics[width=0.44\textwidth]{colours.pdf}}
\caption{Quenching timescale $\tau$ versus quenching onset time $t$ in all three panels. Colour shadings show model predictions of the $u-r$ optical colour (top panel), $NUV-u$ colour (middle panel), and star formation rate in $M_\odot \rm{~yr}^{-1}$ (lower panel), at $t^{obs} = 12.8~\rm{Gyr}$, the mean `observed' time of the GZ2 sample. The combination of optical and NUV colours is a sensitive measure of the $\theta = [t_q, \tau]$ parameter space. Note that all models with $t > 12.8$ \rm{Gyr} are effectively un-quenched. The 'kink' in the bottom panel is due to the assumption that the sSFR is constant prior to $t \sim 3~\rm{Gyr}$ ($z\sim 2.2$).}
\label{pred}
\end{figure}

The quenched star formation history (SFH) of a galaxy can be simply modelled as an exponentially declining star formation rate (SFR) across cosmic time ($0 \leq t ~\rm{[Gyr]} \leq 13.8$) as:
\begin{equation}\label{sfh}
SFR =
\begin{cases}
i_{sfr}(t_q) & \text{if } t < t_q \\
i_{sfr}(t_q) \times exp{\left( \frac{-(t-t_{q})}{\tau}\right)} & \text{if } t > t_q 
\end{cases}
\end{equation}
where $t_{q}$ is the onset time of quenching, $\tau$ is the timescale over which the quenching occurs and $i_{sfr}$ is an initial constant star formation rate dependent on $t_q$.  A smaller $\tau$ value corresponds to a rapid quench, whereas a larger $\tau$ value corresponds to a slower quench. 

We assume that all galaxies formed at a time $t=0~\rm{Gyr}$ with an initial burst of star formation. The mass of this initial burst is controlled by the value of the $i_{sfr}$ which is set as the average sSFR at the time of quenching $t_q$. \citet{Peng} defined a relation (their equation 1) by {\changed empirically fitting to SDSS data for the average specific star formation rate (sSFR) and redshift (cosmic time, $t$)} as:
\begin{equation}
sSFR(m,t) = 2.5 \left( \frac{m}{10^{10} M_{\odot}} \right)^{-0.1} \left(\frac{t}{3.5}\right)^{-2.2} \rm{Gyr}^{-1}.
\end{equation}
Beyond $z \sim 2$ the characteristic SFR flattens and is roughly constant back to $z\sim6$. The cause for this change is not well understood but can be seen across similar observational data \citep{Peng, Gonzalez, Beth}. Motivated by these observations, the relation defined in \citet{Peng} is taken up to a cosmic time of $t=3~\rm{Gyr}~(z \sim 2.3)$ and prior to this a constant average SFR is assumed (see Figure~\ref{sfr_mass_col}). At the point of quenching, $t_{q}$, the models are defined to have a SFR which lies on this relationship for the sSFR, for a galaxy with mass, $m = 10^{10.27} M_{\odot}$ (the mean mass of the GZ2 sample; see Section~\ref{results}).
  
Under these assumptions the average SFR of our models will result in a lower value than the relation defined in \citet{Peng} at all cosmic times with this treatment; each galaxy only resides on the `main sequence' at the point of quenching. However galaxies cannot remain on the `main sequence' from early to late times throughout their entire lifetimes given the unphysical stellar masses and SFRs this would result in at the current epoch in the local Universe \citep{Beth, Heinis14}. If we were to include prescriptions for no quenching, starbursts, mergers, AGN etc. into our models we would improve on our reproduction of the average SFR across cosmic time; however we chose to initially focus on the simplest model possible.

Once this evolutionary SFR is obtained, it is convolved with the \citet{BC03} population synthesis models to generate a model SED at each time step. The observed features of galaxy spectra can be modelled using simple stellar population techniques which sum the contributions of individual, coeval, equal-metallicity stars. The accuracy of these predictions depends on the completeness of the input stellar physics. Comprehensive knowledge is therefore required of (i) stellar evolutionary tracks and (ii) the initial mass function (IMF) to synthesise a stellar population accurately. 

These stellar population synthesis (SPS) models are an extremely well explored (and often debated) area of astrophysics \citep{Maraston05, Eminian08, CGW09, Falk09, Chen10, Kriek10, MRC11, Mel12}. In this investigation we chose to utilise the \citet{BC03} \emph{GALEXEV} SPS models, to allow a direct comparison with S14, along with a Chabrier \citep{Chab03} IMF, across a large wavelength range ($0.0091 < ~\lambda~\rm{[\mu m]}~ < 160 $) with solar metallically (m62 in the \citet{BC03} models).

%{\newchange This choice of SPS model can incorporate systematic uncertainties into the results, but the \citet{BC03} stellar population synthesis models are a frequent choice for many investigations of galaxy evolution, such as \citet{Bundy06, deLucia07, Daddi07, Salim07, Kewley08}.  A comparison between results produced with two or more SPS models is beyond the scope of this work.} %therefore the BC03 models} were chosen over the \citet{Maraston05} models due to their proven reliability in reproducing colours of `typical' galaxies. \citet{Maraston05} focus on improving the implementation of TP-AGB stars in these models, which are known to severely redden post-starburst populations \citep{MG07, Kriek10}. Given that we are exploring a quenching star formation history model, rather than a starburst, we believe that the \citet{BC03} models are more appropriate for this investigation. 

Fluxes from stars younger than $3~$Myr in the SPS model are suppressed to mimic the large optical depth of protostars embedded in dusty formation cloud (as in S14), then filter transmission curves are applied to the fluxes to obtain AB magnitudes and therefore colours. {\changed Given that information about these modelled populations is available across all cosmic time, they can be `observed' at a given time in their history $t^{obs}$; this corresponds with the observed redshift of the GZ2 sample given the modelling assumptions.} Therefore, for the GZ2 sample, the observed redshift was used to calculate the assumed age of each galaxy, in order to compare the observed colours to the predicted models colours directly. 

Figure~\ref{pred} shows these predicted optical and NUV colours at a time of $t^{obs} = 12.8 ~\rm{Gyr}$ (the average observed time of the Galaxy Zoo 2 sample, $z \sim 0.076$) provided by the exponential SFH model. These predicted colours will be referred to as $d_{c,p}(t_{q}, \tau, t^{obs})$, where c=\{opt,NUV\} and p = predicted. The SFR at a time of $t^{obs}=12.8~\rm{Gyr}$ is also shown in Figure~\ref{pred} to compare how this correlates with the predicted colours. The $u-r$ predicted colour shows an immediate correlation with the SFR, however the $NUV-u$ colour is more sensitive to the value of $\tau$ and so is ideal for tracing any recent star formation in a population . At small $\tau$ (rapid quenching timescales) the $NUV-u$ colour is insensitive to $t_{q}$, whereas at large $\tau$ (slow quenching timescales) the colour is very sensitive to $t_{q}$. Together the two colours are ideal for tracing the effects of $t_{q}$ and $\tau$ in a population. 

{\newchange We stress here that this model is not a fully hydrodynamical simulation, it is a simple model built in order to test the understanding of the evolution of galaxy populations. These models are therefore not expected to accurately determine the SFH of every galaxy in the GZ2 sample, in particular galaxies which have not undergone any queching. In this case the models described above can only attribute a constant star formation rate to these  unquenched galaxies. In reality, there are many possible forms of SFH that a galaxy can take, a few of which have been investigated in previous literature; starburts \citep{Canalizo01}, a power law \citep{Glazebrook03}, single stellar populations \citep{Trager00, Sanchez06, Vazdekis10} and metallicity enrichment \citep{deLucia14}. Incorporating these different SFHs into our models is a possible future extension to this work once the results of this initial study are well enough understood to permit additional complexity to be added.}  

%however an investigation including such a range of SFHs is beyond the scope of this study.}


\subsection{Probabilistic Fitting}\label{stats}


In order to achieve robust conclusions we conduct a Bayesian analysis \citep{Sivia, MacKay} of our SFH models in comparison to the observed GZ2 sample data. This approach requires consideration of all possible combinations of $\theta \equiv (t_{q}, \tau)$. Assuming that all galaxies formed at $t=0~\rm{Gyr}$ with an initial burst of star formation, we can assume that the `age' of each galaxy in the GZ2 sample is equivalent to an observed time, $t^{obs}_{k}$ (see Section~\ref{class}). We then use this  `age' to calculate the predicted model colours at this cosmic time for a given combination of $\theta$: $d_{c,p}(\theta_k, t^{obs}_{k})$ for both optical and NUV $(c={opt,NUV})$ colours. We can now directly compare our model colours with the observed GZ2 galaxy colours, so that for a single galaxy $k$ with optical ($u-r$) colour, $d_{opt, k}$ and NUV ($NUV-u$) colour, $d_{NUV,k}$, the {\changed likelihood $P(d_{k}|\theta_k, t^{obs}_{k})$ is}:


\begin{multline}\label{like}
P(d_{k}|\theta_k, t^{obs}_{k}) = \frac{1}{\sqrt{2\pi\sigma_{opt, k}^2}}\frac{1}{\sqrt{2\pi\sigma_{NUV, k}^2}} \\ \exp{\left[ - \frac{(d_{opt, k} - d_{opt, p}(\theta_k, t_{k}^{obs}))^2}{\sigma_{opt, k}^2} \right]} \\ \exp{\left[ - \frac{(d_{NUV, k} - d_{NUV, p}(\theta_k, t_{k}^{obs}))^2}{\sigma_{NUV, k}^2} \right]},
\end{multline}


We have assumed that $P(d_{opt}|\theta_k, t^{obs}_{k})$ and $P(d_{NUV}|\theta_k, t^{obs}_{k})$ are independent of each other and that the errors on the observed colours are also independent. To obtain the probability of each combination of $\theta$ values \underline{given} the GZ2 data: $P(\theta_k|d_k, t^{obs})$, i.e. how likely is a single SFH model given the observed colours of a single GZ2 galaxy, {\changed we utilise Bayes' theorem}:
{\changed \begin{equation}\label{big}
P(\theta_k|d_k) = \frac{P(d_k|\theta_k, t^{obs})P(\theta_k)}{\int P(d_k |\theta_k, t^{obs})P(\theta_k) d\theta_k}.
\end{equation}}
{\changed We assume a flat prior on the model parameters so that:
\begin{equation}\label{prior}
P(\theta_k) =
\begin{cases}
1 & \text{if } 0 \leq t_q ~\rm{[Gyr]}~ \leq 13.8 ~  \text{ and } ~ 0 \leq \tau  ~\rm{[Gyr]}~ \leq 4\\
0 & \text{otherwise} \\
\end{cases}
\end{equation}}

\begin{figure}
\centering{
\includegraphics[width=0.5\textwidth]{triangle_t_tau_red_s_1237655504035185152_40000_14_16_06_08_14.pdf}}
\caption{{\changed Example output from \starfpy ~for a galaxy within the red sequence. The contours show the positions of the `walkers' in the Markov Chain (which are analogous to the areas of high probability) for the quenching models described by $\theta = (t_q, \tau)$ and the histograms show the 1D projection along each axis. Solid (dashed) blue lines show the best fit model (with $\pm 1\sigma$) to the galaxy data. The postage stamp image from SDSS is shown in the top right along with the vote fractions for smooth ($p_s$) and disc ($p_d$) from Galaxy Zoo 2.} }
\label{one_example}
\end{figure}

As the denominator of Equation~\ref{big} is a normalisation factor, comparison between likelihoods for two different SFH models (i.e., two different combinations of $\theta_k = [t_q, \tau]$) is equivalent to a comparison of the numerators. Calculation of $P(\theta_k|d_k, t^{obs})$  for any $\theta$ is possible given data for the GZ2 sample (or a sub-sample thereof). Markov Chain Monte Carlo (MCMC; \citealt{MacKay, Dan, GW10}) provides a robust comparison of the likelihoods between $\theta$ values; here we choose a Python implementation of an affine invariant ensemble sampler by \cite{Dan}; \emph{emcee}.

This method allows for a more efficient exploration of the parameter space by avoiding those areas with low likelihood. A large number of `walkers' are started at an initial position where the likelihood is calculated; from there they individually `jump' to a new area of parameter space. If the likelihood in this new area is greater (less) than the original position then the `walkers' accept (reject) this change in position. Any new position then influences the direction of the  `jumps' of other walkers.  {\changed This is repeated for the defined number of steps after an initial 'burn-in' phase. \emph{emcee} returns the positions of these `walkers', which are analogous to the regions of high probability in the model parameter space.} The model outlined above has been coded using the \emph{Python} programming language into a package named \starfpy ~which has been made freely available to download\footnote{{\changed github.com/rjsmethurst/starfpy}}. {\changed An example output from this Python package for a single galaxy from the GZ2 sample in the red sequence is shown in Figure~\ref{one_example}. The contours show the positions of the `walkers' in the Markov Chain which are analogous to the areas of high probability.}

{\changed We wish to consider the model parameters for the populations of galaxies across the colour magnitude diagram for both smooth and disc galaxies, therefore we run the \starfpy ~package on each galaxy in the GZ2 sample. This was extremely time consuming; for each combination of $\theta$ values which \emph{emcee} proposes, a new SFH must be built, prior to convolving it with the BC03 SPS models at the observed age and then predicted colours calculated from the resultant SED. For a single galaxy this takes up to 2 hours on a typical desktop machine for long Markov Chains. A look-up table was therefore generated at $50 ~t^{obs}$, for $100 ~t_{quench}$ and $100 ~\tau$ values; this was then interpolated over for a given observed galaxy's age and proposed $\theta$ values at each step in the Markov Chain. This ensured that a single galaxy took 2 minutes to run on a typical desktop machine. This interpolation was found to incorporate an error of $\pm 0.04$ into the median $\theta$ values (see Appendix section~\ref{app_lookup} for further information). 

Using this lookup table, each of the $126,316$ total galaxies in the GZ2 sample was run through \starfpy ~on multiple cores of a computer cluster to obtain the Markov Chain positions (analogous to $P(\theta_k|d_k)$) for each galaxy, $k$ (see Figure~\ref{one_example}). In each case the Markov Chain consisted of $100$ `walkers' which took $400$ steps in the `burn-in' phase and $400$ steps thereafter, at which point the MCMC acceptance fraction was checked to be within the range $0.25 < f_{acc} < 0.5$ (which was true in all cases). {\newchange Due to the Bayesian nature of this method, a statistical test on the results is not possible; the output is probabilistic in nature across the entirety of the parameter space.} }

\begin{figure*}
\centering{
\includegraphics[width=0.9\textwidth]{sfr_mass_subsets.pdf}}
\caption{Star formation rate versus stellar mass diagrams show how the different populations of galaxies  (top row, left to right: all galaxies, GZ2 `clean' disc and smooth galaxies; bottom row, left to right: blue cloud, green valley and red sequence galaxies) contribute to the star forming sequence (from \citet{Peng}, shown by the solid blue line with 0.3 dex scatter by the dashed lines). Based on positions in these diagrams, the green valley does appear to be a transitional population between the blue cloud and the red sequence. Detailed analysis of star formation histories can elucidate the nature of the different populations' pathways through the green valley. The clean smooth and disc samples are described in Section~\ref{class}.}
\label{sfr_mass_sub}
\end{figure*}

\begin{figure*}
\centering{
\includegraphics[width=\textwidth]{sfr_mass_colour_diagram.pdf}}
\caption{Left panel: SFR vs. $M_*$ for all 126,316 galaxies in our full sample (shaded contours), with model galaxy trajectories shown as coloured points/lines with each point representing a time step of $0.5~\rm{Gyr}$. The SFHs of the models are shown in the middle panel, where the SFR is initially constant before quenching at time $t$ and thereafter exponentially declining with a characteristic timescale $\tau$. We set the SFR at the point of quenching to be consistent with the typical SFR of a star-forming galaxy at the quenching time, $t$ (dashed line; \citealt{Peng}). The full range of models reproduces the observed colour-colour properties of the sample (right panel); for clarity the figures show only 4 of the possible models explored in this study.}
\label{sfr_mass_col}
\end{figure*}

{\newchange These individual galaxy positions are then combined to visualise the areas of high probability in the model parameter space across a given population (e.g. the green valley).} We do this by first discarding positions with a corresponding probability of $P(\theta_k|d_k) < 0.2$ in order to exclude galaxies which are not well fit by the quenching model; for example blue cloud galaxies which are still star forming will be poorly fit by a quenching model (see Section~\ref{sfh}). The Markov Chain positions are then binned and weighted by their {\newchange corresponding logarithmic probability $\log [P(\theta_k|d_k)]$, provided by the \emph{emcee} package, to further emphasise the features and differences between each population}. The GZ2 data also provides uniquely powerful continuous measurements of a galaxy's morphology, therefore we utilise the user vote fractions to obtain separate model parameters for both smooth and disc galaxies. This is obtained by also weighting by the morphology vote fraction when the binned positions are summated. {\newchange We stress that this portion of the methodology is a non-Bayesian visualisation of the combined individual galaxy results for each population.}

For example, the galaxy shown in Figure~\ref{one_example} would contribute almost evenly to both the smooth and disc parameters due to the GZ2 vote fractions. Since galaxies with similar vote fractions contain both a bulge and disc component, this method is effective in incorporating intermediate galaxies which are thought to be crucial to the morphological changes between early- and late-type galaxies. It was the consideration of these intermediate galaxies which was excluded from the investigation by S14.

%, as follows:
%\begin{equation}\label{sum}
%P(\theta_{disc} | \underline{d}) = \frac{1}{Z} \sum_k p_{disc, k} P(\theta_k|d_k),
%\end{equation}
%and
%\begin{equation}\label{sum2}
%P(\theta_{smooth} | \underline{d}) = \frac{1}{Z} \sum_k p_{smooth, k} P(\theta_k|d_k), 
%\end{equation}
%where Z is a normalisation constant. 



\section{Results}\label{results}
\subsection{Initial Results} 

\begin{figure*}
\includegraphics[width=0.4975\textwidth]{red_smooth.pdf}
\includegraphics[width=0.4975\textwidth]{red_disc.pdf}
\caption[8pt]{{\changed Contour plots showing the combined positions in the Markov Chain for all galaxies in the red sequence, weighted by the logarithmic probability of each position (see Section \ref{stats}) and also by the morphological vote fractions from GZ2 to give the areas of high probability in the model parameter space for both bulge (left) and disc (right) dominated systems. The histograms show the projection into one dimension for each parameter. The dashed lines show the separation between rapid ($\tau ~\rm{[Gyr]} < 1.0$), intermediate ($1.0 < \tau ~\rm{[Gyr]} < 2.0$) and slow ($\tau ~\rm{[Gyr]} > 2.0$) quenching timescales with the fraction of the combined posterior probability distribution in each region shown (see Section~\ref{stats}).}}
\label{red_s}
\end{figure*}


Figure~\ref{sfr_mass_sub} shows the SFR versus the stellar mass for the observed GZ2 sample which has been split into blue cloud, green valley and red sequence populations as well as into the `clean' disc and smooth galaxy samples (with GZ2 vote fractions of $p_d \geq 0.8$ and $p_s \geq 0.8$ respectively). The green valley galaxies are indeed a population which have either left, or begun to leave, the star forming sequence or have some residual star formation still occurring. Interestingly, when we compare those galaxies which reside on the star forming sequence we find that the tail ends of this population (the low and high mass  galaxies) are primarily made up of smooth galaxies as opposed to disc galaxies.


%\begin{figure*}
%\centering{
%\includegraphics[width=\textwidth]{sfr_mass_evo.pdf}}
%\caption{Top left panel shows the SFR plotted against $M_{*}$ for the sample of GZ2 galaxies used in this investigation. The rest of the panels show the evolution of this diagram as predicted by the models at different observed times in the history of the Universe (top row, left to right: $z\sim0$, $z\sim0.07$; bottom row, left to right: $z\sim0.35$, $z\sim0.6$, $z\sim1$). The SFR vs. mass relation from \citet{Peng} for a galaxy of mass $M=10^{10.27} M_{\odot}$ at the corresponding epoch is shown in all panels; we do not expect to reproduce this relation as our models focus on the quenched galaxies below this.}
%\label{sfr_mass_evo}
%\end{figure*}

The left panel in Figure~\ref{sfr_mass_col} shows how well the quenching models reproduce the observed relationship between the SFR and the mass of a galaxy, including how at the time of quenching they reside on the star forming sequence shown by the solid black line for a galaxy of mass, $M = 10^{10.27} M_{\odot}$.  %We can also see in Figure~\ref{sfr_mass_evo} how this SFR vs. mass diagram evolves with cosmic time according to our model predictions. We expect to not reproduce the relationship between the SFR and mass (as shown by the dashed lines from \citealt{Peng} in all panels) with our models as they focus entirely on quenched galaxies. However we can see how our model predicts the build up of galaxies below the SFR vs. mass relation for star forming galaxies across cosmic time so that by the current epoch ($t\sim13.8~\rm{Gyr}$ in the top middle panel of Figure~\ref{sfr_mass_evo}) we can see that it provides a good match to that observed in the GZ2 sample (top left panel of Figure~\ref{sfr_mass_evo}).

%The contour plots in Figures~\ref{all}-~\ref{blue_c_clean} show the regions of high likelihood for the SFH model parameters $\theta = (t, \tau)$ for both smooth- and disc-like galaxies (left and right panels respectively) when considering the different subsets of the galaxies in the GZ2 sample. These plots were produced by \starfpy ~using the output of the MCMC sampling method outlined in Section~\ref{stats}. A `burn-in' of 100 steps was utilised initially and then run until the `walkers' had converged to one or more likelihoods peaks. The histograms show the distribution for the individual parameters and are normalised to the same value across all the Figures in this Section. The colours in the background are provided as a reference to the predicted SFR at the average look-back time of the GZ2 sample ($t^{obs}=12.8~\rm{Gyr}$, see Figure~\ref{pred}). 

{\changed These models were implemented in the \starfpy ~package to produce Figures~\ref{red_s},~\ref{green_v} \&~\ref{blue_c} for the red sequence, green valley and blue cloud populations of smooth and disc galaxies respectively.} In this Section we refer to rapid, intermediate and slow quenching timescales which correspond to ranges of {\changed $\tau ~\rm{[Gyr]} < 1.0$, $1.0 < \tau ~\rm{[Gyr]} < 2.0$ and $\tau ~\rm{[Gyr]} > 2.0$} respectively. 



\subsection{Red Sequence Galaxies}\label{rs}


The left hand panel of Figure~\ref{red_s} reveals that smooth galaxies found in the optical red sequence {\changed show a preference $(49.5\%)$} for rapid quenching timescales across all cosmic time. These are therefore the typical \emph{`red and dead'} galaxies expected to be found on the red sequence. 

{ \changed The rapid quenching timescales are dominant especially in smooth galaxies to $z \sim 1$, prior to which intermediate and slower quenching timescales are the preferred mode. If we take into account the observed peak of average star formation rate in the Universe, which has been found to occur at $z\sim2$ \citep{Hopkins04}, we can assume that when more SF is occurring, quenching is a more arduous task. We can see that post $z\sim2$ the preference for the quenching time begins to rise (top histogram in the left hand panel of Figure~\ref{red_s}), coinciding with the decline in the average star formation.}

{\changed For these smooth red sequence galaxies we also see, at early times only, a preference for slow and intermediate timescales in the left panel of Figure~\ref{red_s}. Perhaps this is the influence of intermediate galaxies (with $p_s \sim p_d \sim 0.5$), hence why similar high likelihood areas exist for both the smooth-like and disc-like galaxies}. This is especially apparent considering there are far more of these intermediate galaxies than those that are definitively early- or late-types (see Table~\ref{subs}). These galaxies are those whose morphology cannot be easily distinguished either because they are at a large distance or because they are an S0 galaxy whose morphology can be interpreted by different users in different ways. \citet{GZ2} find that S0 galaxies expertly classified by \citet{NA10} are more commonly classified as ellipticals by GZ2 users, but have a significant tail to high disc vote fractions, giving a possible explanation as to the origin of this area of probability.

%This suggests that since these smooth galaxies quench at either rapid or intermediate timescales before these times, that the majority of them reside on the red sequence by $t \ga 8~\rm{Gyr}$. This is consistent with the findings of \citet{Im02} who, using data from the Deep Groth Strip Survey (GSS), found that the majority of elliptical galaxies at the current epoch ($\ga 70\%$) already existed at $z\sim1$ and have not undergone any dramatic evolution since then. 

{\changed The right panel of Figure~\ref{red_s} reveals that disc galaxies show similar preferences for rapid $(31.3\%)$ and slow $(44.1\%)$ quenching timescales. The preference for \emph{very} slow ($\tau > 3.0 ~\rm{Gyr}$) quenching timescales (the likes of which are not seen in either the green valley or blue cloud, see Figures~\ref{green_v} and~\ref{blue_c})} suggests that these  galaxies have only just reached the red sequence after a very slow evolution across the colour-magnitude diagram. Considering their limited number it is likely that these galaxies are currently on the edge of the red sequence having recently (and finally) moved out of the green valley. Table~\ref{subs} shows that $3.9\%$ of our sample are red sequence clean disc galaxies, i.e. red late-type spirals. This is, within uncertainties, in agreement with the findings of \citet{Masters10}, who find $\sim6\%$ of late-type spirals are red when defined by a cut in the $g-r$ optical colour at the `blue end of the red sequence' (rather than with $u-r$ as implemented in this investigation).

%{\changed The disc galaxies also show equal preference for rapid as for slow quenching timescales, however only at recent times, which is contradictory to the findings of S14. This could again be due to the intermediate galaxies, however since the percentages in Figure~\ref{red_s} are similar for rapid and slow quenching timescales, unlike in the smooth galaxies. This suggests some other underlying mechanism that affects these disc galaxies. 

{\newchange Despite the dominance of slow quenching timescales, the red sequence disc galaxies also show some preference for rapid quenching timescales ($31.3\%$), similar to the red sequence smooth galaxies but with a lower probability. Perhaps these rapid quenching timescales can also be attributed to a morphological change, suggesting that the quenching has occurred more rapidly than the morphological change to a bulge dominated system.}

The top histogram in the right hand panel shows that the quenching preference for disc galaxies is distributed evenly over cosmic time. Downsizing has only ever been observed in elliptical galaxies \citep{Thomas10} therefore this detection of constant occurrence of quenching confirms that irrespective of the mechanism, the evolution of discs is constant over cosmic time. 

Comparing the resultant SFRs for both the smooth- and disc-like galaxies in Figure~\ref{red_s} by noticing {\changed where the areas of high probability lie with respect to the bottom panel of Figure~\ref{pred} (which shows the predicted SFR at an observation time of $t\sim12.8~\rm{Gyr}$, the average `observed' time of the GZ2 population)} reveals that  red sequence disc galaxies with a preference for slow quenching still have some residual star formation occurring, SFR$~\sim0.105 M_{\odot}yr^{-1}$, whereas the smooth galaxies with a dominant preference for rapid quenching have a resultant SFR$~\sim0.0075 M_{\odot}yr^{-1}$. This is approximately 14 times less than the residual SFR still occurring in the red sequence disc galaxies. Within error, this is in agreement with the findings of \citet{Toj13} who, by using the VErsatile SPectral Analyses spectral fitting code {\newchange (VESPA; \citealt{Tojero07})}, found that red late-type spirals show 17 times more recent star formation than red elliptical galaxies.

These results for red sequence galaxies have many implications for green valley galaxies, as all of these systems must have passed through the green valley on their way to the red sequence. %Therefore if the green valley is a transitional population we expect to see a similar distribution of likelihoods for the green valley disc-like and smooth-like galaxies but perhaps at later quenching times, pre-empting what will become the red sequence over time. 

\subsection{Green Valley Galaxies}\label{gv}

\begin{figure*}
\includegraphics[width=0.4975\textwidth]{green_smooth.pdf}
\includegraphics[width=0.4975\textwidth]{green_disc.pdf}
\caption[8pt]{{\changed Contour plots showing the combined positions in the Markov Chain for galaxies in the green valley, weighted by the logarithmic probability of each position (see Section \ref{stats}) and also by the morphological vote fractions from GZ2 to give the areas of high probability in the model parameter space for both bulge (left) and disc (right) dominated systems. The histograms show the projection into one dimension for each parameter. The dashed lines show the separation between rapid ($\tau ~\rm{[Gyr]} < 1.0$), intermediate ($1.0 < \tau ~\rm{[Gyr]} < 2.0$) and slow ($\tau ~\rm{[Gyr]} > 2.0$) quenching timescales with the fraction of the combined posterior probability distribution in each region shown (see Section~\ref{stats}).}}
\label{green_v}
\end{figure*}

In Figure~\ref{green_v} we can make similar comparisons for the green valley galaxies to those discussed previously for the red sequence. {\changed Among green valley galaxies the preference for rapid quenching timescales drops by over a factor of two for both morphologies. {\newchange For the red sequence galaxies, an argument can be made for two possible tracks across the green valley, shown by the bimodal nature of both distributions in $\tau$ with a common area in the intermediate timescales region where the rapid and slow timescales peaked distributions intersect. However in the green valley this intermediate quenching timescale region becomes more significant, particularly for the smooth-like galaxies (see the left panel of Figure~\ref{green_v}). }

The smooth galaxy parameters favour these intermediate quenching timescales ($40.6\%$) with some preference for slow quenching at  early times ($z > 1$).} This may be counter intuitive at first, as one of the main arguments for the lack of galaxies in the green valley is due to the hypothesised rapid movement across it; however the proportion of present day green valley galaxies with this history will therefore be lower. {\changed Those galaxies with such a rapid decline in star formation will pass so quickly through the green valley they will be detected at a lower number than those galaxies which have stalled in the green valley with intermediate quenching timescales;} accounting for the observed number of intermediate galaxies which are present in the green valley {\newchange and the dominance of rapid timescales detected in the red sequence for both morphologies.} 

{\changed The disc galaxy parameters for the green valley population now overwhelmingly prefer slow quenching timescales ($47.4\%$) with a similar amount of intermediate quenching to the smooth galaxy parameters ($37.6\%$; see Figure~\ref{green_v}).} Compared to red sequence disc galaxies, there is still some preference for galaxies with a star formation history which results in a high current SFR, suggesting there are also some late-type galaxies that have just progressed from the blue cloud into the green valley. 

{\changed If we compare Figure~\ref{green_v} to Figure~\ref{red_s} we can see a larger amount of quenching is occurring at later (more recent) cosmic times in the green valley than in the red sequence for both galaxy types.} Therefore both morphologies are tracing the evolution of the red sequence, confirming that the green valley is indeed a transitional population between blue cloud and red sequence regardless of morphology. Currently as we observe the green valley, its main constituent are very slowly evolving disc-like galaxies along with intermediate- and smooth-like galaxies which pass across it with intermediate timescales within $\sim 1.0-1.5~\rm{Gyr}$.

Given enough time ($t\sim4 - 5~\rm{Gyr}$), the disc galaxies will eventually fully pass through the green valley and make it out to the red sequence (the right panel of Figure~\ref{sfr_mass_col} shows galaxies with $\tau > 1.0~\rm{Gyr}$ do not approach the red sequence within $3~\rm{Gyr}$ post quench). This is most likely the origin of the `red spirals'.

{\changed If we consider the red sequence as the `end point' of galactic evolution, then we can expect that the ratio of smooth:disc galaxies that is currently observed in the green valley will evolve into the ratio observed in the red sequence. Table~\ref{subs} shows the ratio of smooth : disc galaxies in the observed red sequence of the GZ2 sample is $62:38$ whereas in the green valley it is $45:55$. In order to obtain the same ratio as seen in the red sequence, $31.2\%$ of the disc-dominated galaxies currently residing in the green valley would have to undergo a morphological change to a bulge-dominated galaxy. We find that the fraction of the probability for green valley disc galaxies occupying the parameter space $\tau < 1.5 ~\rm{Gyr}$ is $29.4\%$, and therefore suggest that quenching mechanisms with these timescales are capable of destroying the disc-dominated nature of galaxies.}

All of this evidence suggests that there are not just two routes for galaxies through the green valley but a continuum of quenching time scales which we can divide into three regimes:: {\newchange with rapid ($\tau < 1.0 ~\rm{Gyr}$), intermediate ($1.0 < \tau < 2.0~\rm{Gyr}$) and slow ($\tau > 2.0~\rm{Gyr}$). The intermediate quenching time scale galaxies live in the space between the extremes sampled by the UV/optical diagrams of S14; the inclusion of the intermediate galaxies in this investigation (unlike in S14) and the more precise Bayesian analysis, quantifies this range of $\tau$ and specifically ties the intermediate timescales to all variations of galaxy morphology.} %Since S14 did not consider intermediate galaxies, their conclusions that there are two routes through the green valley for galaxies contradicts with our statements that their are three or more routes, dependant on whether a galaxy is early-, intermediate- or late-type. 


\subsection{Blue Cloud Galaxies}\label{bc}

\begin{figure*}
\includegraphics[width=0.4975\textwidth]{blue_smooth.pdf}
\includegraphics[width=0.4975\textwidth]{blue_disc.pdf}
\caption[8pt]{{\changed Contour plots showing the combined positions in the Markov Chain for galaxies in the blue cloud, weighted by the logarithmic probability of each position (see Section \ref{stats}) and also by the morphological vote fractions from GZ2 to give the areas of high probability in the model parameter space for both bulge (left) and disc (right) dominated systems. The histograms show the projection into one dimension for each parameter. The dashed lines show the separation between rapid ($\tau ~\rm{[Gyr]} < 1.0$), intermediate ($1.0 < \tau ~\rm{[Gyr]} < 2.0$) and slow ($\tau ~\rm{[Gyr]} > 2.0$) quenching timescales with the fraction of the combined posterior probability distribution in each region shown (see Section~\ref{stats}). Positions with probabilities less than 0.2 are discarded as poorly fit models, therefore we can conclude unsurprisingly that blue cloud galaxies are not well described by a quenching star formation model. }}
\label{blue_c}
\end{figure*}

Since the blue cloud is considered to be primarily made of star forming galaxies we expect \starfpy to have some difficulty in determining the most likely quenching model to describe them, as confirmed by Figure~\ref{blue_c}. The attempt to characterise a star forming galaxy with a quenched SFH model leads \starfpy to attribute the extremely blue colours of the majority of these galaxies to fast quenching at recent times (i.e. very little change in the SFR; see the right hand panel of Figure~\ref{blue_c} in comparison with the bottom panel of Figure~\ref{pred}).

{\changed Similarly the only quenching detected within the blue cloud disc galaxy population is for rapid quenching at recent times. Perhaps even galaxies which are currently quenching slowly across the the blue cloud cannot be well fit by the quenching models implemented, as they still have high SFRs despite some quenching (although a galaxy has undergone quenching, star formation can still occur in a galaxy, just at a slower rate than at earlier times, described by $\tau$).} 


{\changed There is a very small preference among blue bulge dominated galaxies for slow quenching which began prior to $z \sim 0.5 $. These populations have been blue for a considerable period of time, slowly using up their gas for star formation by the Kennicutt$-$Schmidt law \citep{Schmidt59, Kennicutt97}. However the major preference is for rapid quenching at recent times in the blue cloud; this therefore supports the theories for blue ellipticals as either merger-driven ($\sim76\%$) or gas inflow-driven reinvigorated star formation that is now slowly decreasing ($\sim24\%$).}

The blue cloud is therefore primarily composed of both star forming disc or smooth galaxies and smooth galaxies which {\changed are undergoing a rapid quench, presumably after a previous event triggered star formation and turned them blue}. 



\section{Discussion}\label{diss}

We have implemented a Bayesian statistical analysis of the star formation histories (SFHs) of a large sample of galaxies morphologically classified by Galaxy Zoo. We have found differences between the SFHs of smooth- and disc-like galaxies across the colour-magnitude diagram in the red sequence, green valley and blue cloud. In this section we will speculate on the question: what are the possible mechanisms driving this difference and the mechanisms responsible for the three different timescales observed? 

\subsection{Rapid Quenching Mechanisms}\label{rapid}

{\changed Rapid quenching is much more prevalent in smooth galaxies than disc galaxies, and red sequence galaxies are also much more likely to be characterised by a rapid quenching model than green valley galaxies (ignoring blue cloud galaxies due to their apparent poor fit by the quenching models, see Figure~\ref{blue_c}). In the green valley there is also a distinct lack of preference for rapid quenching timescales with $\tau < 0.5~\rm{Gyr}$.} This suggests that this rapid quenching mechanism causes a change in morphology from a disc- to a smooth-like galaxy as it quickly traverses the colour-magnitude diagram to the red sequence, {\newchange supported by the number of disc galaxies that would need to undergo a morphological change in order for the disc:smooth ratio of the galaxies in the green valley to match that of the red sequence (see Section~\ref{gv})}. {\newchange From this indirect evidence we suggest that} this rapid quenching mechanism is due to major mergers.

% In Sections~\ref{rs},~\ref{green_v} and~\ref{bc} we find rapid quenching occurring at quenching times $t\sim4~\rm{Gyr}$, $t\sim11~\rm{Gyr}$ and $t\sim13.8~\rm{Gyr}$ for the red sequence, green valley and clean blue cloud galaxies respectively.

Inspection of the galaxies contributing to this area of likelihood reveals that this does not arise due to \emph{currently} merging pairs identified by GZ users, but by typical smooth galaxies with red optical and NUV colours that the model attributes to rapid quenching at early times.

One simulation of interest by \citet{Springel05} showed that feedback from black hole activity is a necessary component of destructive major mergers to produce such rapid quenching timescales. Powerful quasar outflows remove much of the gas from the inner regions of the galaxy, terminating star formation on extremely short timescales. \citet{Bell06}, using data from the COMBO-17 redshift survey ($0.4 < z < 0.8$), estimate a merger timescale from being classified as a close galaxy pair and recognisably disturbed as $\sim 0.4~\rm{Gyr}$. \citet{Springel05} consequently find using hydrodynamical simulations that after $\sim1~\rm{Gyr}$ the merger remnant has reddened to $u-r \sim 2.0$. This is in agreement with our simple quenching models which show (Figure~\ref{sfr_mass_col}) that within $\sim1~\rm{Gyr}$ the models with a SFH with $\tau < 0.4~\rm{Gyr}$ have reached the red sequence with $u-r ~\ga 2.2$. {\changed This could explain the preference for red sequence disc galaxies with rapid quenching timescales ($31.3\%$), as they may have undergone a major merger recently but are still undergoing a morphological change from disc, to disturbed, to an eventual smooth galaxy (see also \citealt{vdW09}).} 

%We can also compare our predictions for the fractions of clean red sequence galaxies which have undergone such a rapid quenching star formation history, to the predicted merger fraction from observations and simulations. We find that of the galaxies in our sample that are defined as both clean and residing within the red sequence, $\sim4.2\%$ of them have a preferred likelihood for models with $\tau < 0.4~\rm{Gyr}$. This is in agreement with the observations by \citet{Bell06} who showed that $5\% \pm 1\%$ of massive galaxies (with $M_* > 2.5 \times 10^{10} M_{\odot}$) are in a close galaxy pair and of \citet{Bundy09} who found that the fraction of systems resulting in a major merger was $4\%$, and that this fraction did not change significantly with redshift. 

%Most of this rapid quenching occurs for smooth galaxies in the red sequence (see Figure~\ref{red_s_clean}) and it occurs before a quenching time of $t\sim6~\rm{Gyr}$ ($z\sim1$). This is in agreement with \citet{Im02} who state that if major mergers are responsible for the formation of luminous ellipticals (and S0s) such a process must have occurred predominantly before ($z\sim1$). 

{\changed We reiterate that this rapid quenching mechanism occurs much more rarely in green valley galaxies of both morphologies than in the red sequence, however does not fully characterise all the galaxies in either the red sequence or green valley.} Dry major mergers therefore do not fully account for the formation of any galaxy type at any redshift, supporting the observational conclusions made by \citet{Bell07,Bundy07, Kav14} and simulations by \citet{Genel08}.

\subsection{Intermediate Quenching Mechanisms}\label{int}
{\changed Intermediate quenching timescales are found to be equally prevalent across populations for both smooth and disc galaxies across cosmic time, {\newchange particularly in the green valley. Unlike in the red sequence, intermediate timescales are the prevalent mechanism for quenching smooth green valley galaxies, unlike the rapid quenching prevalent in the red sequence.}} We suggest that this intermediate quenching route must therefore be possible with routes that both preserve and transform morphology. It is this result of {\newchange another route through the green valley that is in contradiction} with the findings of S14. 

If we once again consider the simulations of \citet{Springel05}, this time without any feedback from black holes, they suggest that if even a small fraction of gas is not consumed in the starburst following a merger (either because the mass ratio is not large enough or from the lack of strong black hole activity) the remnent can sustain star formation for periods of several Gyrs. The remnants from these simulations take $\sim5.5~\rm{Gyr}$ to reach red optical colours of $u-r \sim 2.1$. We can see from Figure~\ref{sfr_mass_col} that the models with intermediate quenching timescales of $1.0 \la ~\tau~\rm{[Gyr]} ~\la 2.0$ take approximately $2.5-5.5~\rm{Gyr}$ to reach these red colours.

We speculate that the intermediate quenching timescales are caused by gas rich major mergers, major mergers without black hole feedback and from minor mergers, the latter of which is the dominant mechanism. This is supported by the findings of \citet{Lotz08}  who find that the timescales for equal mass gas rich mergers with large initial separations range from $\sim 1.1-1.9~\rm{Gyr}$, and of \citet{Lotz11}, who find in further simulations that as the baryonic gas fraction in a merger with mass ratios of 1:1-1:4 increases, so does the timescale of the merger from $\sim0.2~\rm{Gyr}$ (with little gas, as above for major mergers causing rapid quenching timescales) up to $\sim1.5~\rm{Gyr}$ (with large gas fractions). 

The simulations in \citet{Lotz08} also show that the remnants of these equal mass gas rich disc mergers (wet disc mergers) are observable for $\ga1~\rm{Gyr}$ post merger and appear ``disc-like and dusty", which is consistent with an \emph{early-type spiral morphology}.  Such galaxies are often observed to have spiral features with a dominant bulge, suggesting that such galaxies may divide the votes of the GZ2 users, producing vote fractions of $p_s \sim p_d \sim 0.5$. We believe this is why the intermediate quenching timescales are equally dominant for both smooth and disc galaxies across each population in Figures~\ref{red_s} and~\ref{green_v}. 

Other simulations (e.g. such as \citet{Rob06} and \citet{Barnes02}) support the conclusion that both gas rich major mergers and minor mergers can produce disc-like remnants. Observationally, \citet{Darg10a} showed an increase in the spiral to elliptical ratio for merging galaxies ($0.005 < z < 0.1$) by a factor of two compared to the general population. They attribute this to the much longer timescales during which mergers of spirals are observable compared to mergers with elliptical galaxies, {\changed confirming our hypothesis that the quenching timescales $\tau < 1.5 ~\rm{Gyr}$ preferred by disc galaxies may be undergoing mergers which will eventually lead to a morphological change}. Similarly, \citet{Casteels13} observe that galaxies ($0.01 < z < 0.09$) which are interacting often retain their spiral structures and that a spiral galaxy which has been classified as having `loose winding arms' by the GZ2 users are often entering the early stages of mergers and interactions.

{\changed $40.6\%$ of the likelihood for smooth galaxies in the green valley arises due to intermediate quenching timescales (see Figure~\ref{green_v}); this is agreement with work done by \citet{Kav14} who by studying SDSS photometry ($z<0.07$) state that approximately half of the star formation in early-type galaxies is driven by minor mergers at $0.5 < z < 0.7$ therefore exhausting available gas for star formation and consequently causing a gradual decline in the star formation rate}. This supports earlier work by \cite{Kav11} who, using multi wavelength photometry of galaxies in COSMOS \citep{Scoville07}, found that $70\%$ of early-type galaxies  appear morphologically disturbed, suggesting either a minor or major merger in their history. {\changed This is in agreement with the total percentage of probability with $\tau < 2.0 ~\rm[Gyr]$; $73.9\%$ and $59.3\%$, for the smooth red sequence and green valley galaxies in Figures~\ref{red_s} and~\ref{green_v} respectively.}

The resultant intermediate quenching timescales occur due to one interaction mechanism occurring, unlike the rapid quenching, which occurs due to a major merger combined with AGN feedback, and decreases the SFR over a short period of time. Therefore any external event which can cause either a burst of star formation (depleting the gas available) or directly strip a galaxy of its gas (for example galaxy harassment, interactions, ram pressure stripping and interactions internal to clusters) would cause an quenching on an intermediate timescale. Considering that the majority of galaxies reside in groups or clusters where such interactions are common, it is not surprising that the majority of our galaxies are considered intermediate in morphology (see Table~\ref{subs}) {\newchange and therefore are undergoing or have undergone such an interaction.}


\subsection{Slow Quenching Timescales}\label{slow}
Although intermediate and rapid quenching timescales are the dominant mechanisms across the colour-magnitude diagram, together they cannot completely account for the quenching of disc galaxies. S14 concluded that slow quenching timescales were the most dominant mechanism for disc galaxies. {\changed However we show that: (i) intermediate quenching timescales are equally important in the green valley and (ii) rapid quenching timescales are equally important in the red sequence.} There is also a significantly lower preference for smooth galaxies to undergo such slow quenching timescales; suggesting that the evolution (or indeed creation) of typical smooth galaxies is dominated by processes external to the galaxy. {\changed This is excepting galaxies in the blue cloud where a small amount of slow evolution of blue ellipticals is occurring, presumably after a reinvigoration of star formation which is slowly depleting the gas available according to the Kennicutt$-$Schmidt law.}

\citet{Bamford09} using GZ1 vote fractions of galaxies in the SDSS, found a significant fraction of high stellar mass red spiral galaxies in the field. As these galaxies are isolated from the effects of interactions from other galaxies, the slow quenching mechanisms present in their preferred star formation histories are most likely due to secular processes (i.e. mechanisms internal to the galaxy, in the absence of sudden accretion or merger events; \citealt{KK04, Sheth12}). Bar formation in a disc galaxy is such a mechanism, whereby gas is funnelled to the centre of the galaxy by the bar over long timescales where it is used for star formation {\newchange \citep{Masters12, Saint12, Cheung13}}, consequently forming a `pseudo-bulge' \citep{Kormendy10, Simmons13}.

 If we believe that these slow quenching timescales are due to secular evolution processes, this is to be expected since these processes do not change the disc dominated nature of a galaxy. 

\subsection{Future Work}\label{future}
Due to the flexibility of our model we believe that the \starfpy ~module will have a significant number of future applications, including the investigation of various different SFHs (e.g. constant SFR and starbursts). Considering the number of magnitude bands available across the SDSS, further analysis will also be possible with a larger set of optical and NUV colours, providing further constraints. It would also be of interest to consider galaxies at higher redshift {\changed (e.g. out to $z \sim 1$ with Hubble Space Telescope photometry and the GZ:Hubble project, see \citealt{Melvin14} for first results) and consider different redshift bins in order to study the build up of the red sequence with cosmic time. }

With further use of the robust, detailed GZ2 classifications, we believe that \starfpy will be able to distinguish any statistical difference in the star formation histories of barred vs. non-barred galaxies. This will require a simple swap of $\{p_s, p_d\}$ with $\{p_{bar}, p_{no bar}\}$ from the available GZ2 vote fractions. We believe that this will aid in the discussion of whether bars act to quench star formation (by funnelling gas into the galaxy centre) or promote star formation (by causing an increase in gas density as it travels through the disc) both sides of which have been fiercely argued \citep{Masters11, Masters12, Sheth05, Ellison11}. 

%Further application of the \starfpy ~code could be to investigate the parameters for currently merging/interacting pairs to those galaxies classified as merger remnants from their degree of disturbance. This will allow a direct comparison of the impact of a merger on the star formation rate of a galaxy by comparing the before and after scenario's. 
%
%It also thought that depending on the initial conditions of a galaxy merger this can either produce a slow or fast rotator elliptical galaxy. Since slow rotators are on average more massive objects that they may result from major mergers (dry) on the red sequence \citep{Em11} , whereas fast rotators are obtained in simulations from gas rich (wet) mergers and can form more disc-like objects \citep{Em07}. By using a sample of fast and slow rotators as input galaxies we may even be able to detect a difference in the star formation histories of wet and dry mergers and in turn how these two separate populations of elliptical galaxies evolved. {\newchange The SAMI \citep{SAMI} and MANGA \citep{MANGA} surveys will provide large samples of galaxies in order to test these ideas in future work.}
%
%One of the key elements of an overarching theory of galaxy evolution is to explain how field and cluster galaxies evolve in comparison to each other. The projected neighbour density, $\Sigma$, from \citet{Baldry06} can be used to weight the probabilities of cluster and field galaxies (instead of the vote fractions from GZ2 for $\{p_s, p_d\}$) to determine if there is a measurable statistical difference in their star formation histories. 
{\newchange Further application of the \starfpy ~code could be to investigate the SFH parameters of:
\begin{enumerate}
\item Currently merging/interacting pairs in comparison to those galaxies classified as merger remnants, from their degree of morphological disturbance,
\item Slow rotators and fast rotators which are thought to result from dry major mergers on the red sequence \citep{Em11} and gas rich, wet major mergers \citep{Em07} respectively,
\item Field and cluster galaxies using the projected neighbour density, $\Sigma$, from \citet{Baldry06}.
\end{enumerate}
The SAMI \citep{SAMI} and MANGA \citep{MANGA} surveys will provide large samples of galaxies in order to test these ideas in future work.}



\section{Conclusion}\label{conc}
We have used morphological classifications from the Galaxy Zoo 2 project to determine the morphology-dependent star formation histories of galaxies via a Bayesian analysis of an exponentially declining star formation quenching model. We determined the most likely parameters for the quenching time, $t_q$ and quenching timescale $\tau$ in this model for galaxies across the blue cloud, green valley and red sequence to trace galactic evolution across the colour-magnitude diagram. We find that the green valley is indeed a transitional population for all morphological types (in agreement with \citet{Sch2014}), however this transition proceeds slowly for the majority of disc-like galaxies and occurs rapidly for the majority of smooth-like galaxies in the red sequence. However, in addition to \citet{Sch2014}, {\changed our Bayesian approach has revealed a more nuanced result, specifically that the prevailing mechanism across all morphologies and populations is quenching with intermediate timescales}. Our main findings are as follows:
\begin{enumerate}
\item Red sequence galaxies, regardless of morphology, are found to have similar preferences for quenching timescales to the green valley galaxies, but occurring at earlier quenching times (see Figures~\ref{red_s} and~\ref{green_v}). Therefore the quenching mechanisms currently occurring in the green valley were active in creating the red sequence at earlier times; confirming that the green valley is indeed a transitional population, regardless of morphology.

\item {\changed The typical red sequence galaxy is elliptical in morphology and has undergone a rapid to intermediate quench at any cosmic time, resulting in a very low current SFR (see Section~\ref{rs}).}

\item {\changed The green valley as it is currently observed is dominated by very slowly evolving disc-like galaxies along with intermediate- and smooth-like galaxies which pass across it with intermediate timescales within $\sim 1.0-1.5~\rm{Gyr}$ (see Section~\ref{gv}).}

\item There are three routes through the green valley dependant on galaxy type, with the smooth- and disc-like galaxies each having different dominant star formation histories across the colour-magnitude diagram. 

\item {\changed Blue cloud galaxies are not well fit by a quenching model of star formation due to the continuous high star formation rates occurring (see Figure~\ref{blue_c}).}

\item {\changed The distribution for the quenching time for disc galaxies is found to be constant over all redshifts, confirming the assumption that regardless of the mechanism, the evolution of disc galaxies is constant over time.}

\item {\changed Rapid quenching timescales are detected with a lower probability for green valley galaxies than red sequence galaxies.} We speculate that this quenching mechanism is caused by major mergers with black hole feedback, which are able to expel the remaining gas not initially exhausted in the merger-induced starburst and which can cause a change in morphology from disc- to bulge-dominated. The colour-change timescales from previous simulations of such events agree with our derived timescales  {\changed (see Section~\ref{rapid}). These rapid timescales are instrumental in forming the red sequence, however galaxies at the current epoch passing through the green valley do so at more intermediate timescales (see Figure~\ref{green_v}).}

\item Intermediate quenching timescales ($1.0 < ~\tau~\rm{[Gyr]}~ < 2.0 $) are found with constant probability across the red sequence and green valley for both smooth- and disc-like morphologies, the timescales for which agree with observed and simulated minor merger timescales (see Section~\ref{int}). We hypothesise such timescales can be caused by a number of external processes, including gas rich major mergers, mergers without black hole feedback, galaxy harassment, interactions and ram pressure stripping. The timescales and observed morphologies from previous studies agree with our findings, including that this is the dominant mechanisms for intermediate galaxies such as early-type spiral galaxies with spiral features but a dominant bulge, which split the GZ2 vote fractions (see Section~\ref{int}). 

\item Slow quenching timescales are the most dominant mechanism in the disc galaxy populations across the colour-magnitude diagram. Disc galaxies are often found in the field, therefore we hypothesise that such slow quenching timescales are caused by secular evolution and processes internal to the galaxy. {\changed We also detect a small amount of slow quenching timescales for blue elliptical galaxies which we attribute to a reinvigoration of star formation which is slowly depleting the gas available (see Section~\ref{bc}).}

\item Due to the flexibility of this model we believe that the \starfpy ~module compiled for this investigation will have a significant number of future applications, including the different star formation histories of barred vs non-barred galaxies, merging vs merger remnants, fast vs slow rotating elliptical galaxies and cluster vs field galaxies (see Section~\ref{future}).
\end{enumerate}

%There is not one specific route to each location of the colour-magnitude diagram, due to the complex interplay between the SFH parameters, however the morphology of a galaxy can reveal the previous possible steps to reach it's current location.

\section*{Acknowledgements}
The authors would like to thank D. Forman-Mackey for extremely useful Bayesian statistics discussions and J. Binney for an interesting discussion on the nature of quenching and feedback in disc galaxies. 

RS acknowledges funding from the Science and Technology Facilities Council Grant Code ST/K502236/1. BDS gratefully acknowledges support from the Oxford Martin School, Worcester College and Balliol College, Oxford. KS gratefully acknowledges support from Swiss National Science Foundation Grant PP00P2\_138979/1. KLM acknowledges funding from The Leverhulme Trust as a 2010 Early Career Fellow. TM acknowledges funding from the Science and Technology Facilities Council Grant Code ST/J500665/1. KWW and LF acknowledge funding from a Grant-in-Aid from the University of Minnesota.

The development of Galaxy Zoo was supported in part by the Alfred P. Sloan Foundation. Galaxy Zoo was supported by The Leverhulme Trust. 

Based on observations made with the NASA Galaxy Evolution Explorer.  GALEX is operated for NASA by the California Institute of Technology under NASA contract NAS5-98034

Funding for the SDSS and SDSS-II has been provided by the Alfred P. Sloan Foundation, the Participating Institutions, the National Science Foundation, the U.S. Department of Energy, the National Aeronautics and Space Administration, the Japanese Monbukagakusho, the Max Planck Society, and the Higher Education Funding Council for England. The SDSS Web Site is http://www.sdss.org/.
The SDSS is managed by the Astrophysical Research Consortium for the Participating Institutions. The Participating Institutions are the American Museum of Natural History, Astrophysical Institute Potsdam, University of Basel, University of Cambridge, Case Western Reserve University, University of Chicago, Drexel University, Fermilab, the Institute for Advanced Study, the Japan Participation Group, Johns Hopkins University, the Joint Institute for Nuclear Astrophysics, the Kavli Institute for Particle Astrophysics and Cosmology, the Korean Scientist Group, the Chinese Academy of Sciences (LAMOST), Los Alamos National Laboratory, the Max-Planck-Institute for Astronomy (MPIA), the Max-Planck-Institute for Astrophysics (MPA), New Mexico State University, Ohio State University, University of Pittsburgh, University of Portsmouth, Princeton University, the United States Naval Observatory, and the University of Washington.

This publication made extensive use of the Tool for Operations on Catalogues And Tables (TOPCAT; ~\citealt{Taylor05}) which can be found at \url{http://starlink.ac.uk/topcat/}. Ages were calculated from the observed redshifts using the \emph{cosmolopy} package provided in the Python module \emph{astroPy}\footnote{\url{http://www.astropy.org/}}; \citealt{Rob13}). This research has also made use of NASA's ADS service and Cornell's ArXiv. 

\begin{thebibliography}{}
\bibitem[\protect\citeauthoryear{Aihara et al.}{2011}]{Aihara11} Aihara, H. et al., 2011, ApJSS, 193, 29
\bibitem[\protect\citeauthoryear{Arnouts et al.}{2007}]{Arnouts07} Arnouts, S. et al., 2007, A\&A, 476, 137
\bibitem[\protect\citeauthoryear{Baldry et al.}{2004}]{Baldry04} Baldry, I. K. et al., 2004, ApJ, 600, 681
\bibitem[\protect\citeauthoryear{Baldry et al.}{2006}]{Baldry06} Baldry, I. K. et al., 2006, MNRAS, 373, 469
\bibitem[\protect\citeauthoryear{Ball, Loveday \& Brunner}{2008}]{BLB08} Ball, N. M., Loveday, J. \& Brunner, R. J., 2008, MNRAS, 383, 907
\bibitem[\protect\citeauthoryear{Bamford et al.}{2009}]{Bamford09} Bamford, S. P. et al., 2009, MNRAS, 393, 1324
\bibitem[\protect\citeauthoryear{Barnes \& Hernquist}{1996}]{BH96} Barnes, J. E. \& Hernquist, L., 1996, ApJ, 471, 115
\bibitem[\protect\citeauthoryear{Barnes}{2002}]{Barnes02} Barnes, J. E., 2002, MNRAS, 333, 481
\bibitem[\protect\citeauthoryear{Bell et al.}{2004}]{Bell04} Bell, E. F. et al., 2004, ApJ, 608, 752
\bibitem[\protect\citeauthoryear{Bell et al.}{2006}]{Bell06} Bell, E. F. et al., 2006, ApJ, 652, 270
\bibitem[\protect\citeauthoryear{Bell et al.}{2007}]{Bell07} Bell, E. F. et al., 2007, ApJ, 663, 834
\bibitem[\protect\citeauthoryear{B\'ethermin et al.}{2012}]{Beth} B\'ethermin, M. et al., 2012, ApJ, 757, L23
\bibitem[\protect\citeauthoryear{Blanton et al.}{2005}]{Blanton05} Blanton, M. R. et al., 2005, AJ, 129, 2562
\bibitem[\protect\citeauthoryear{Blanton \& Roweis}{2007}]{BR07} Blanton, M. R. \& Roweis, S., 2007, AJ, 133, 734
\bibitem[\protect\citeauthoryear{Brammer et al.}{2009}]{Brammer09} Brammer, G. B. et al., 2009, ApJ, 706, 173
\bibitem[\protect\citeauthoryear{Brinchmann et al.}{2004}]{Brinch04} Brinchmann, J. et al., 2004, MNRAS, 351, 1151
\bibitem[\protect\citeauthoryear{Bruzual \& Charlot}{2003}]{BC03} Bruzual, G. \& Charlot, S., 2003, MNRAS, 344, 1000
\bibitem[\protect\citeauthoryear{Bundy et al.}{2006}]{Bundy06} Bundy, K. et al., 2006, ApJ, 651, 120
\bibitem[\protect\citeauthoryear{Bundy et al.}{2007}]{Bundy07} Bundy, K. et al., 2007, ApJL, 655, L5
\bibitem[\protect\citeauthoryear{Bundy et al.}{2009}]{Bundy09} Bundy, K. et al., 2009, ApJ, 697, 1369
\bibitem[\protect\citeauthoryear{Bundy et al.}{2010}]{Bundy10} Bundy, K. et al., 2010, ApJ, 719, 1969
\bibitem[\protect\citeauthoryear{Bundy et al.}{in prep.}]{MANGA} Bundy, K. et al., \emph{in preparation}
\bibitem[\protect\citeauthoryear{Canalizo \& Stockton}{2001}]{Canalizo01} Canalizo, G. \& Stockon, A., 2001, ApJ, 555, 719
\bibitem[\protect\citeauthoryear{Cardelli et al.}{1989}]{Cardelli89} Cardelli, J. A. et al., 1989, ApJ, 345, 245
\bibitem[\protect\citeauthoryear{Casteels et al.}{2013}]{Casteels13} Casteels, K. et al., 2013, MNRAS, 429, 1051
\bibitem[\protect\citeauthoryear{Chabrier et al.}{2003}]{Chab03} Chabrier, G., 2003, PASP, 115, 763
\bibitem[\protect\citeauthoryear{Chen et al.}{2010}]{Chen10} Chen, X. Y. et al., 2010, A\&A, 515, 101
\bibitem[\protect\citeauthoryear{Cheung et al.}{2013}]{Cheung13} Cheung, E. et al., 2013, ApJ, 779, 162
\bibitem[\protect\citeauthoryear{Conroy, Gunn \& White}{2009}]{CGW09} Conroy, C., Gunn, J. E. \& White, M. 2009, ApJ, 699, 486
\bibitem[\protect\citeauthoryear{Croom et al.}{2012}]{SAMI} Croom, S. et al. 2012, MNRAS, 421, 872
\bibitem[\protect\citeauthoryear{Daddi et al.}{2007}]{Daddi07} Daddi, E. et al., 2007, ApJ, 670, 156
\bibitem[\protect\citeauthoryear{Darg et al.}{2010}]{Darg10a} Darg, D. et al., 2010a, MNRAS, 401, 1552
\bibitem[\protect\citeauthoryear{de Lucia \& Blaizot}{2007}]{deLucia07} de Lucia, G. \& Blaizot, J., 2007, MNRAS, 375, 2
\bibitem[\protect\citeauthoryear{de Lucia}{2014}]{deLucia14} de Lucia, G., 2014, (arXiv:1407:7867)
\bibitem[\protect\citeauthoryear{Ellison et al.}{2011}]{Ellison11} Ellison, S. L. et al., 2001, MNRAS, 416, 2182
\bibitem[\protect\citeauthoryear{Emsellem et al.}{2007}]{Em07} Emsellem, E. et al., 2007, IAU Symposium 235
\bibitem[\protect\citeauthoryear{Emsellem et al.}{2011}]{Em11} Emsellem, E, et al., 2011, MNRAS, 414, 888
\bibitem[\protect\citeauthoryear{Eminian et al.}{2008}]{Eminian08} Eminian, C. et al., 2008, MNRAS, 384, 930
\bibitem[\protect\citeauthoryear{Faber et al.}{2007}]{Faber07} Faber, S. M. et al., 2007, ApJ, 665, 265
\bibitem[\protect\citeauthoryear{Falomo et al.}{2008}]{Falomo08} Falomo, R. et al., 2008, ApJ, 673, 694
\bibitem[\protect\citeauthoryear{Falkenberg et al.}{2009}]{Falk09} Falkenberg, M. A. et al., 2009, MNRAS, 397, 1954
\bibitem[\protect\citeauthoryear{Foreman-Mackey et al.}{2013}]{Dan} Foreman-Mackey, D., Hogg, D. W., Lang, D., Goodman, J., 2013, PASP, 125, 306
\bibitem[\protect\citeauthoryear{Genel et al.}{2008}]{Genel08} Genel, S. et al., 2008, ApJ, 688, 789
\bibitem[\protect\citeauthoryear{Glazebrook et al.}{2003}]{Glazebrook03} Glazebrook, K. et al., 2003, ApJ, 587, 55
\bibitem[\protect\citeauthoryear{Goodman \& Weare}{2010}]{GW10} Goodman, J. \& Weare, J., 2010, CAMCS, 5, 65
\bibitem[\protect\citeauthoryear{Gon\c calves et al.}{2012}]{Gonc12} Gon\c calves, T. S. et al., 2012, ApJ, 759, 67
\bibitem[\protect\citeauthoryear{Gonz\'alez et al.}{2010}]{Gonzalez} Gonz\'alez, V. et al., 2010, ApJ, 713, 115
\bibitem[\protect\citeauthoryear{Heinis et al.}{2014}]{Heinis14} Heinis, S. et al., 2014, MNRAS, 437, 1268
\bibitem[\protect\citeauthoryear{Hopkins}{2004}]{Hopkins04} Hopkins, A. M., 2004, ApJ, 615, 209
\bibitem[\protect\citeauthoryear{Im et al.}{2002}]{Im02} Im, M. et al., 2002, ApJ, 571, 136
\bibitem[\protect\citeauthoryear{Jarosik et al.}{2011}]{WMAP} Jarosik, N. et al., 2011, ApJSS, 192, 18
\bibitem[\protect\citeauthoryear{Kauffmann et al.}{2003}]{Kauff03} Kauffman, G. et al., 2003, MNRAS, 341, 33
\bibitem[\protect\citeauthoryear{Kaviraj et al.}{2011}]{Kav11} Kaviraj, S. et al. 2011, MNRAS, 411, 2148
\bibitem[\protect\citeauthoryear{Kaviraj}{2014}]{Kav14} Kaviraj, S., 2014, MNRAS, 440, 2944
\bibitem[\protect\citeauthoryear{Kennicutt}{1997}]{Kennicutt97} Kennicutt, R. C., 1997, ApJ, 498, 491
\bibitem[\protect\citeauthoryear{Kewley \& Ellison}{2008}]{Kewley08} Kewley, L. J. \& Ellison, S. L., 2008, ApJ, 681, 1183
\bibitem[\protect\citeauthoryear{Kormendy \& Kennicutt}{2004}]{KK04} Kormendy, J. \& Kennicutt, R. J., 2004, ARA\&A, 42, 603
\bibitem[\protect\citeauthoryear{Kormendy et al.}{2010}]{Kormendy10} Kormendy, J. et al., 2010, ApJ, 723, 54
\bibitem[\protect\citeauthoryear{Kriek et al.}{2010}]{Kriek10} Kriek, M. et al., 2010, ApJL, 722, L64
\bibitem[\protect\citeauthoryear{Lintott et al.}{2009}]{Lintott09} Lintott, C. J. et al., 2009, MNRAS, 389, 1179
\bibitem[\protect\citeauthoryear{Lintott et al.}{2011}]{Lintott11} Lintott, C. J. et al., 2011, MNRAS, 410, 166
\bibitem[\protect\citeauthoryear{Lotz et al.}{2008}]{Lotz08} Lotz, J. et al., 2008, MNRAS, 391, 1137
\bibitem[\protect\citeauthoryear{Lotz et al.}{2011}]{Lotz11} Lotz, J. et al., 2011, MNRAS, 742, 103
\bibitem[\protect\citeauthoryear{MacKay}{2003}]{MacKay} MacKay, D. J. C., 2003, \emph{Information Theory, Inference and Learning Algorithms}, Cambridge University Press, ISBN 978-0-521-64298-9
\bibitem[\protect\citeauthoryear{Marasco, Fraternali \& Binney}{2012}]{MFB12} Marasco, A., Fraternali, F. \& Binney, J. J., 2012, MNRAS, 419, 1107
\bibitem[\protect\citeauthoryear{Maraston}{2005}]{Maraston05} Maraston, C., 2005, MNRAS, 362, 799
\bibitem[\protect\citeauthoryear{Marigo \& Girardi}{2007}]{MG07} Marigo, P. \& Girardi, L. 2007, A\&A, 469, 239
\bibitem[\protect\citeauthoryear{Martin et al.}{2005}]{Martin05} Martin, D. C. et al., 2005, ApJ, 619, L1
\bibitem[\protect\citeauthoryear{Martin et al.}{2007}]{Martin07} Martin, D. C. et al., 2007, ApJS, 173, 342
\bibitem[\protect\citeauthoryear{Masters et al.}{2010}]{Masters10} Masters, K. L. et al., 2010, MNRAS, 405, 783
\bibitem[\protect\citeauthoryear{Masters et al.}{2011}]{Masters11} Masters, K. L. et al., 2011, MNRAS, 411, 2026
\bibitem[\protect\citeauthoryear{Masters et al.}{2012}]{Masters12} Masters, K. L. et al., 2012, MNRAS, 424, 2180
\bibitem[\protect\citeauthoryear{Melbourne et al.}{2012}]{Mel12} Melbourne, J. et al., 2012, ApJ, 748, 47
\bibitem[\protect\citeauthoryear{Melvin et al.}{2014}]{Melvin14} Melvin, T. et al., 2014, MNRAS, 438, 2882
\bibitem[\protect\citeauthoryear{Mendez et al.}{2011}]{Mendez11} Mendez, A. J. et al., 2011, ApJ, 736, 110
\bibitem[\protect\citeauthoryear{Miller, Rose \& Cecil}{2011}]{MRC11} Miller, N. A., Rose, J. A. \& Cecil, G. 2011, ApJL, 727, L15
\bibitem[\protect\citeauthoryear{Nair \& Abraham}{2010}]{NA10} Nair, P. B. \& Abraham, R. G. 2010, ApJSS, 186, 427 
\bibitem[\protect\citeauthoryear{Noeske et al.}{2007}]{Noeske07} Noeske, K. G. et al., 2007, ApJ, 660, L43
\bibitem[\protect\citeauthoryear{Oh et al.}{2011}]{Oh11} Oh, K. et al., 2011, ApJS, 195, 13
\bibitem[\protect\citeauthoryear{Padmanabhan et al.}{2008}]{Pad08} Padmanabhan, N. et al., 2008, ApJ, 674, 1217
\bibitem[\protect\citeauthoryear{Peng et al.}{2010}]{Peng} Peng, Y. et al., 2010, ApJ, 721, 193
\bibitem[\protect\citeauthoryear{Pan et al.}{2014}]{Pan14} Pan, Z. et al., 2014, ApJL, 792, L4
\bibitem[\protect\citeauthoryear{Robertson et al.}{2006}]{Rob06} Robertson, B. et al., 2006, ApJ, 645, 986
\bibitem[\protect\citeauthoryear{Robitaille et al.}{2013}]{Rob13} Robitaille, T. P. et al., 2013, A\&A, 558, A33
\bibitem[\protect\citeauthoryear{Salim et al.}{2007}]{Salim07} Salim, S. et al., 2007, ApJSS, 173, 267
\bibitem[\protect\citeauthoryear{S�nchez-Bl�zquez et al.}{2006}]{Sanchez06} S�nchez-Bl�zquez, P. et al., 2006, A\&A, 457, 809
\bibitem[\protect\citeauthoryear{Saintonge et al.}{2012}]{Saint12} Saintonge, A. et al., 2012, ApJ, 758, 73
\bibitem[\protect\citeauthoryear{Schawinski et al.}{2007}]{Sch07} Schawinski, et al., 2007, MNRAS, 382, 1415
\bibitem[\protect\citeauthoryear{Schawinski et al.}{2009}]{Sch09} Schawinski, K. et al., 2009, MNRAS, 396, 818
\bibitem[\protect\citeauthoryear{Schawinski et al.}{2014}]{Sch2014} Schawinski, K. et al., 2014 (arXiv: 1402.4814)
\bibitem[\protect\citeauthoryear{Schiminovich et al.}{2007}]{Schim07} Schiminovich, D. et al., 2007, ApJS, 173, 315
\bibitem[\protect\citeauthoryear{Schmidt}{1959}]{Schmidt59} Schmidt, M., 1959, ApJ, 129, 243
\bibitem[\protect\citeauthoryear{Scoville et al.}{2007}]{Scoville07} Scoville, N. et al., 2007, ApJSS, 172, 1
\bibitem[\protect\citeauthoryear{Sheth et al.}{2005}]{Sheth05} Sheth, K. et al., 2005, ApJ, 632, 217
\bibitem[\protect\citeauthoryear{Sheth et al.}{2012}]{Sheth12} Sheth, K. et al., 2012, ApJ, 758, 136
\bibitem[\protect\citeauthoryear{Simmons et al.}{2013}]{Simmons13} Simmons, B. D. et al., 2013, MNRAS, 429, 2199
\bibitem[\protect\citeauthoryear{Sivia}{1996}]{Sivia} Sivia, D. S., 1996, \emph{Data Analysis: A Bayesian Tutorial}, Oxford University Press, ISBN 0-19-851889-7
\bibitem[\protect\citeauthoryear{Skibba et al.}{2009}]{Skibba09} Skibba, R. A. et al., 2009, MNRAS, 399, 966
\bibitem[\protect\citeauthoryear{Springel, Di~Matteo \& Hernquist}{2005}]{Springel05} Springel, V., Di Matteo, T. \& Hernquist, L., 2005, ApJ, 620, L79
\bibitem[\protect\citeauthoryear{Soklakov}{2002}]{Sok02} Soklakov, A. N., 2002, (arXiv:math-ph/0009007)
\bibitem[\protect\citeauthoryear{Strateva et al.}{2001}]{Strat01} Strateva, I. et al., 2001, AJ, 122, 1861
\bibitem[\protect\citeauthoryear{Taylor}{2005}]{Taylor05} Taylor, M. B., 2005, ASP Conference Series, 347
\bibitem[\protect\citeauthoryear{Thomas et al.}{2010}]{Thomas10} Thomas, D. et al., 2010, MNRAS, 404, 1775
\bibitem[\protect\citeauthoryear{Tojeiro et al.}{2007}]{Tojero07} Tojero, R. et al., 2007, MNRAS, 381, 1252
\bibitem[\protect\citeauthoryear{Tojeiro et al.}{2013}]{Toj13} Tojeiro, R. et al., 2013, MNRAS, 432, 359
\bibitem[\protect\citeauthoryear{Trager et al.}{2000}]{Trager00} Trager, S. C. et al., 2000, AJ, 120, 165
\bibitem[\protect\citeauthoryear{van der Wel et al.}{2009}]{vdW09} van der Wel, A. et al., 2009, ApJ, 706, L120
\bibitem[\protect\citeauthoryear{Vazdekis et al.}{2010}]{Vazdekis10} Vazdekis, A. et al., 2010, MNRAS, 404, 1639
\bibitem[\protect\citeauthoryear{Willett et al.}{2013}]{GZ2} Willett, K. et al., 2014, MNRAS, 435, 2835
\bibitem[\protect\citeauthoryear{Willmer et al.}{2006}]{Willmer06} Willmer, C. N. A. et al., 2006, ApJ, 647, 853
\bibitem[\protect\citeauthoryear{Wong et al.}{2012}]{Wong12} Wong, I. et al., 2012, MNRAS, 420, 1684
\bibitem[\protect\citeauthoryear{Wyder et al.}{2007}]{Wyder07} Wyder, T. K. et al., 2007, ApJS, 173, 293
\bibitem[\protect\citeauthoryear{York et al.}{2000}]{York00} York, D. G. et al., 2000, AJ, 120, 1579
\end{thebibliography}{}

\appendix

\section{Using look up tables}\label{app_lookup}
\begin{figure}
\centering{
\includegraphics[width=0.48\textwidth]{corner_test_starfpy_full_sfh_function_0.pdf}
\includegraphics[width=0.48\textwidth]{corner_test_starfpy_lookup_0.pdf}}
\caption{Top panel: Results from \starfpy ~for \underline{true} $t_q$ and $\tau$ values (red lines) using the complete function to calculate the predicted colour of a proposed set of $\theta$ values in each MCMC iteration. The median value is shown by the solid blue line with the dashed lines encompassing $68\% (\pm 1\sigma)$ of the samples. The time taken to run for a single galaxy using this method is approximately 2 hours. Bottom panel: Results from \starfpy ~for \underline{true} $t_q$ and $\tau$ values using a look up table generated from the complete function to calculate the predicted colour of a proposed set of $\theta$ values in each MCMC iteration. The time taken to run for a single galaxy using this method is approximately 2 minutes.}
\label{lookup}
\end{figure}

{\newchange Considering the size of the sample in this investigation of $126,316$ galaxies total, a three dimensional look up table (in observed time, quenching time and quenching rate) was generated using the star formation history function in \starfpy ~to speed up the run time. Figure~\ref{lookup} shows an example of how using the look up table in place of the full function does not affect the results to a significant level. Table~\ref{median_lu} quotes the median values along with their $\pm 1\sigma$ ranges for both methods in comparison to the true values specified to test \starfpy. The uncertainties incorporated into the quoted values by using the look up table are therefore minimal with a maximum $\Delta = 0.043$.


\begin{table*}
\centering{
\caption{Median values found by \starfpy ~using the complete star formation history function and a look up table to speed up the run time, shown by the blue solid lines in Figure~\ref{lookup}. The errors quoted define the region in which $68\%$ of the samples are located, shown by the dashed blue lines in Figure~\ref{lookup}. The known true values are also quoted, as shown by the red lines in Figure~\ref{lookup}. All values are quoted to three significant figures.}
\begin{tabular*}{0.65\textwidth}{r @{\extracolsep{\fill}}ccc}
\multicolumn{1}{l}{} & \multicolumn{3}{c}{}                                          \\ \hline
                     & $t_q$                       & $\tau$                       &  \\ \hline
True                 & $4.37$                        & $2.12$                         &  \\
Complete             & $3.893 \pm^{3.014}_{2.622}$ & $2.215 \pm^{0.395}_{0.652}$ &  \\
Look up table        & $3.850 \pm^{2.988}_{2.619}$ & $2.218 \pm^{0.399}_{0.649}$ & \\ \hline
\end{tabular*}}
\label{median_lu}
\end{table*}
}
\section{Testing \starfpy}\label{app_test}
{\newchange
In order to test that \starfpy ~can find the correct quenching model for a given observed colour, 25 true (i.e. known) values of $\theta$ were input into \starfpy ~ to ensure that these values were reproduced within error. Figure~\ref{test_mosaic} shows the results of each of these tests, with the known values of $\theta$ shown by the red lines. In some cases this red line does not coincide with the peak of the distribution shown in the histograms for one parameter, however in all cases the intersection of the red lines is within the sample contours. 
\begin{figure*}
\centering{
\includegraphics[width=\textwidth]{mosaic_test.pdf}}
\caption{Results from \starfpy ~for an array of \underline{true} $t_q$ and $\tau$ values (marked by the red lines) using the complete function to calculate the predicted colour of a proposed set of $\theta$ values in each MCMC iteration, assuming an error on the calculated known colours of $\sigma_{u-r} = 0.124$ and $\sigma_{NUV-u} = 0.215$, the average errors on the GZ sample colours. In each case \starfpy ~succeeds in locating the true parameter values within the degeneracies of the star formation history model. These degeneracies can clearly be seen in Figure~\ref{pred}.}
\label{test_mosaic}
\end{figure*}


}


%\appendix
%
%\section{Colour-colour diagram evolution}
%\begin{Figure*}
%\centering{
%\includegraphics[width=\textwidth]{c_c_evo_numpy.pdf}}
%\caption{Plots to show the evolution of the colour-colour diagram as predicted by the most likely exponential SFH model for each GZ2 galaxy. Each panel shows the model at different look-back times in the history of the Universe. The panel on the far right also includes the contours of the observed colours for the GZ2 galaxies in black, as well as the most likely predicted colours, in red, as a comparison.}
%\label{c_c_evo}
%\end{figure*}

\end{document}
